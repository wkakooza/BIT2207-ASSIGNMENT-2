\documentclass[options]{article}
\begin{document}
\textbf{MANNING THE BTS NETWORK }.
\section{\textbf{Abstract:}}.
\section{\textbf{Networking systems of communication are enthusiastic in troubleshooting and manning.}}
\subsection{\textbf{Introduction:}}.
\section{\textbf{What is a BTS?}}. 
\subsection{\textbf{This is a base transceiver station. A Unit used to facilitate wireless communication between user equipment and the network. It is connected to a mobile switching center that enables routing of call between different cells in the network. In other words, it facilitates call setup. 
A user equipment is a unit that records all user call and its responsible for registrations of all users in a given cell radius hence enhancing communications. }}. 
\section{\textbf{How does one Man the BTS network?}}.
\section{\textbf{This is done by regular checking of the BTS in the allocated cell radii to see if the following characteristics are in normal operation;}} 
\subsection{\textbf{1. The station power: if A.C power is on and all equipment are powered and running. 2. The connectivity between different equipment systems at the station. 
This is so easy to do since the tech only comes with a laptop comprising of the system software to help in the connectivity checks and if any errors are sighted, the tech can go on to call for further support in case the predicament is complex or if it a quick fix, then he/she fixes it using the step by step guide(s) of the various equipment’s’ troubleshooting schematics. 
Can everyone man the BTS network? 
I would say yes provided you are interested to learning and adventuring in the networking era. }} 
\section{\textbf{Conclusion:
From the above, urge can be built to learn simple networks and thus join the fun of networking attributes such as manning the BTS network.}}
\section{\textbf{MANNING THE BTS NETWORK 
22/03/2017 
Prepaired by 
Ndagano Robert 
(13/u/22514/eve - 213023231}} 








\end{document}